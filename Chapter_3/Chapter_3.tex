\chapter{The Immersed Boundary Method}
Immersed boundary methods are class of techniques in computational fluid dynamics where the governing equations are solved on cartesian grid that does not conform to the shape of the body in the flow. This is opposed to well known body-conformal techniques where the computational mesh accurately represents the shape of the domain. The boundary condition on the immersed surfaces are not applied explicitly, instead an extra forcing function is added to the governing equations or the discrete numerical scheme is updated near the boundary. The immersed boundary technique is in special interest to us since it removes the mesh sensitivity calculation from the analysis. In this chapter we talk about different classes of immersed boundary technique and apply them to a simple problem. The applicability of these methods in the continuum sensitivity analysis is also discussed. At the end of this chapter, we will chose couple of immersed boundary techniques for sensitivity analysis implementation.

% ======================================================================================
\section{General formulation}
