\chapter{The Immersed Boundary Method}
Immersed boundary methods are class of techniques in computational fluid dynamics where the governing equations are solved on cartesian grid that does not conform to the shape of the body in the flow. This is opposed to well known body-conformal techniques where the computational mesh accurately represents the shape of the domain. The boundary condition on the immersed surfaces are not applied explicitly, instead an extra forcing function is added to the governing equations or the discrete numerical scheme is updated near the boundary. The immersed boundary technique is in special interest to us since it removes the mesh sensitivity calculation from the analysis. In this chapter we talk about different classes of immersed boundary technique and apply them to a simple problem. The applicability of these methods in the continuum sensitivity analysis is also discussed. At the end of this chapter, we will chose couple of immersed boundary techniques for sensitivity analysis implementation.

% ======================================================================================
\section{Introduction}
When people started to use computational models in the design of systems, it was usually sufficient to include single physics into the design. The simulations were usually based on structural solvers using finite elements analysis (FEA) or computational fluid dynamics (CFD) simulations. The design requirements for different systems has been drastically changed compared to the initial designs where these computational methods have been applied. The requirements such as higher fuel efficiency, improved controllability, higher stiffness to mass ratios and lower radar signature have forced the designers to develop more unconventional configurations. For example, on way to reduce the infra-red signature of the aircraft engine is to remove it from hanging underneath the wing and put it inside of the aircraft. However, by doing this as massive heat source will be added the structure. The thermal expansion due to this excessive heat load needs to be incorporated into the structural analysis of the system. It requires a multidisciplinary analysis combining thermal analysis for heat transfer and structural analysis for thermal expansion and other structural loads \cite{deaton2013stiffening}.

The multidisciplinary analysis is required for many engineering application however the one that is used most is the interaction of fluid and a deformable structure. This is commonly known as a Fluid-Solid Interaction (FSI) problem. Fluid–structure interaction (FSI) problems are dealt with in many different engineering applications, such as fluttering and buffeting of bridges \cite{jain1996coupled}, vibration of wind turbine blades \cite{arrigan2011control}, aeroelastic response of airplanes \cite{farhat2006provably}. FSI problems are also seen in blood flows in arteries and artificial heart valves \cite{sotiropoulos2009review}, flying and swimming \cite{kern2006simulations}. The conventional approach for simulating such problems is the Arbitrary Lagrangian–Eulerian (ALE) method. ALE methods are based on body- conforming grids to track the location of the fluid–structure interface. ALE methods have been applied to many FSI problems however, they are cumbersome if not impossible to apply to FSI problems with large deformations for complicated boundary shapes.

Immersed boundary (IB) methods are considered a separate family of methods used for modelling FSI problems with complicated boundary shapes and large deformations. IB methods, are based on solving the governing equations for fluids on a fixed grid. Although this computational grid can be structured (Cartesian) or unstructured, most methods are based on structured grid. When using structured grid, extremely efficient computational methods can be utilized on to solve the governing equations. The fluid–structure boundaries are represented by a set of independent nodes. The solid boundary effect on the flow is formulated either by introducing fictitious body forces in the governing equations or by locally modifying the structure of the background grid.

IB has several advantages over the ALE methods. Probably the biggest advantage is the simplification of the task of grid generation. Generating body-conformal grid for a complex shape is usually very complicated. The objective is to construct a grid that provides adequate local resolution with the minimum number of total grid points.  This requires a significant input from the user and is an iterative process. For complicated boundaries the unstructured grid approach is better suited however the grid quality is reduced for extremely complicated geometry. In contrast, for a simulation carried using an IB method, grid complexity and quality are not affected by the complexity of the geometry. 

This advantage becomes even more clear for flows with moving boundaries. The ALE approach requires generating a new mesh or deforming the old mesh to match the new boundary shape at each time step. The solution from last time step is also required to be projected to this new computational mesh. Both of the deformation/projection can affect the accuracy, robustness and the computational cost associated the the simulation. On the other hand, the boundary motion in IB method can be handled with rather ease because the computational mesh does not depend on the shape of the boundary. Therefore, although a significant progress in simulating flows using ALE methods has been made in the recent years \cite{lomtev1999discontinuous, farhat2004cfd, cheng2005fluid}, the IB method still remains an attractive alternative for such problems due to its simplicity and cost.
