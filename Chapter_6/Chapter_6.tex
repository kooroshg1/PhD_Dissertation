\chapter{Summary Remarks and Future work}
% ======================================================================================
This work is motivated by the need for accurate high fidelity sensitivities for the design of coupled fluid-structure interaction systems. Design and analysis of such multiphysics systems are challenging due to several factors such as the coupling algorithms for data transfer between disciplines, mesh deformation, numerical stabilities, and sensitivity calculation. In this work we focused on two aspects of the FSI problems: i) sensitivity analysis and ii) mesh deformation.

The sensitivity analysis is required in different design aspects such as design optimization, function approximation, and reliability analysis. As pointed out in the literature survey in Chapter \ref{ch:introduction}, the analytical methods for sensitivity calculation have several benefits regarding accuracy and cost over the available numerical methods. However, the primary concern of the analytical sensitivity methods is their formulation and implementation. As discussed in Chapter \ref{ch:sensitivityAnalysis}, there are two approaches for analytical sensitivity calculation, discrete and continuum approach. In the discrete sensitivity analysis (DSA) the governing equations are first discretized and then differentiated whereas the governing equations are first differentiated and then discretized in the continuum sensitivity analysis. It was shown in Chapter \ref{ch:sensitivityAnalysis} that for the CSA the numerical differential operators derived for solving the governing equations can be reused in the sensitivity calculations; however, this is not possible for DSA. The capability of reusing the differential operators without modification is very beneficial concerning the implementation of the CSA in commercial packages. The different between the sensitivity analysis and the solution of the governing equations are through the effect of boundary conditions as was shown in Chapter \ref{ch:sensitivityAnalysis}. The ease of implementation for the CSA motivated us to use this technique for sensitivity calculation of the FSI problem.

The state of the art techniques that are used to model the interaction between the fluid and a deformable solid are based on a body conformal mesh for representing the fluid domain and the solid boundaries. However, as pointed out in Chapter \ref{ch:immersedBoundary} the body-conformal methods require the computational domain to be updated as the solid boundaries deform. This is required for both the design optimization where the optimizer changes the shape and the FSI simulation where the boundaries deform due to forces from the fluid domain. The mesh deformation adds to the computational cost of the simulation since a set of governing equations (usually elastic equations) need to be solved to track the location of the computational nodes. Moreover, these techniques can fail when the deformation of the solid domain becomes significant or for complex boundary shapes. To address these issues, we proposed to utilize the Immersed Boundary (IB) method for solving the flow over solid boundaries and also sensitivity analysis. In IB method, the effect of solid boundaries are either added to the governing equation through force terms (continuous IB method) or represented by changing the numerical stencil near the solid boundaries (discrete method). Several different IB methods were investigated in Chapter \ref{ch:immersedBoundary} and their characteristics were analyzed. The continuous IB method was selected since it is more compatible with CSA. However, additional modifications were made to the continuous IB method to make it appropriate for CSA implementation.

As discussed in Chapter \ref{ch:immersedBoundary}, in the continuum IB method the effect of solid boundaries are added to the governing equations using force terms. This is either through penalizing the governing Navier-Stokes (NS) equations using the Darcy equation or adding the force terms based on a feedback function of Equation \eqref{eq:C3_virtualBoundaryMethod}. These forces are added to the NS equations using the Heaviside function and delta function respectively. However, these functions are not continuously differentiable and cannot be employed in the CSA formulation. The Regularized Delta (RD) and Regularized Heaviside (RH) functions were introduced in Chapter \ref{ch:shapeSenwithIB} to address this issue. This is one of the contributions of this research to make the IB compatible with CSA. The developed RH and RD function are shown in Equations \eqref{eq:C4_heavisideFunction} and \eqref{eq:C4_deltaFunction}. It was shown in Chapter \ref{ch:shapeSenwithIB} that even the results of the IB method improved in accuracy by using the developed RD and RH functions. The sensitivity analysis was conducted for 1D and 2D problems using the IB method in Chapter \ref{ch:shapeSenwithIB} and the sensitivity analysis results were verified using the complex step method. It was shown in Chapter \ref{ch:shapeSenwithIB} that the accuracy of sensitivity results decay near the boundaries. This problem is addressed by reconstructing the sensitivity results using first order Taylor series expansion. The use of CSA alongside the IB method is another contribution of this research. This combines the positive features of IB method regarding robustness and cost to ease of implementation features of CSA. To the best of the author's knowledge, this work is the first implementation of CSA for continuum IB analysis.
% ======================================================================================
\section{Research Contribution}
This dissertation demonstrates several contributions in the area of multidisciplinary sensitivity analysis for the design of systems based on coupled fluid-structure response. This technique offers a significant advantage over the traditional body conformal method for analysis and sensitivity calculation used for design. Some contributions of this work are as follows:
\begin{itemize}
    \item A framework based on immersed boundary formulation for coupled fluid-structure interaction analysis is proposed. By using the immersed boundary method, the need for mesh deformation is removed from the simulation procedure; therefore, a more robust framework is achieved and demonstrated for the simulation of FSI systems where the boundary shape is complex, and the structure deformations are large. The IB method is further improved by the introduction of regularized delta function for transferring data between the fluid and solid domain. This results in a smooth pressure distribution over the solid boundaries and also enables the application of continuum sensitivity analysis for the coupled sensitivity analysis.
    \item The analytical continuum sensitivity formulation is developed for CFD simulations based on the IB approach. The introduction of regularized delta function to the IB formulation enables the continuous differentiation of Navier-Stokes equations. Moreover, by using the continuum sensitivity analysis differential operators used in the solver are utilized without any modification. This methodology enables accurate and rapid sensitivity calculated from pressure and velocity fields of the fluid domain with respect to structure shape. Comparison with complex step results proved the accuracy of sensitivity results calculated with this approach.
    \item A fully coupled multidisciplinary sensitivity analysis is achieved by coupling and differentiating a transient finite element structural solver alongside the CFD simulation. Using this approach, the transient sensitivity response of various coupled fluid-structure interaction systems are investigated. These results can be used in the design of such complex systems.
\end{itemize}

% ======================================================================================
\section{Future Work}
This work proposes a new methodology based on continuum sensitivity analysis and immersed boundary method for the analytical sensitivity analysis of coupled fluid-structure interaction systems. The future accomplishments required in this research are as follows:
\begin{itemize}
    \item In this work, it is assumed that the solid boundary is defined analytically using a couple of control parameters. This enables the analytical differentiation of the boundary shape for the IB method. However, this is not always possible. It is more desired to have a general approach for defining the boundary shape. This is also desirable to the optimization process since arbitrary change becomes possible. We are proposing to use the level-set method for determining the boundaries and calculating the boundary sensitivities required for sensitivity analysis using the IB method. The level set models can easily represent complex boundaries that can form holes, split into multiple pieces, or merge with others to form a single one.
    \item Incorporation of compressible flow simulation is another avenue for the future work in this area. Many of cases such as aircraft design are better represented with the compressible flow assumptions.
    \item All the sensitivity analysis done in this work was based on the direct sensitivity formulation. This formulation is useful where the number of design variables is less than the number of functions that are differentiated. For the case where the functions are less than design variables, the adjoint method is more applicable. The continuous adjoint derivation for implementation using the IB method is required.
\end{itemize}