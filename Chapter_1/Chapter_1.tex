\chapter{Introduction}
% ======================================================================================
\section{Motivation}
Fluid-structure interaction (FSI) problems play important role in many scientific and engineering fields, such as automotive, aerospace, and biomedical industry. Despite the wide application, a comprehensive study of FSI systems still remains a challenge due to their strong nonlinearity and multidisciplinary nature. For most FSI problems, analytical solutions to are not available, and physical experiments are limited in scope. Therefore, to get more insight in the physics involved in the complex interaction between fluids and solids, numerical simulations are used. The numerical solutions are conducted based on Computational Fluid Dynamics (CFD) models for the flow and Finite Element Analysis (FEA) for the structural response. Nevertheless, the prohibitive amount of computations has been one of the major issues in the design optimization of such coupled multidisciplinary systems. The other bottle neck is generating an appropriate computational domain that represents the fluid and solid regions. The effort and time required to take a geometry from a CAD package, clean up the model, and generate a mesh is usually a large portion of the overall human time required for the simulation. This cannot be automated for complex and moving domains. The Immersed Boundary (IB) method, reduces the amount of time needed for the fluid flow simulations and provides fast results by directly addressing the challenges associated with this issue.

Due to the large amount of computations involved in the FSI simulation, the gradient based methods are the best candidates for design optimization of such problems. Sensitivity analysis is the integral part of gradient based methods. Although there are analytical techniques for efficient and accurate sensitivity calculation, they have not yet implemented in commercial CFD packages. Therefore, most gradient optimization techniques relay on finite difference method for sensitivity calculation when solving FSI problem that are prone to errors.

The motivation for the research proposed in this document is in two areas. First, we want to have sensitivity analysis capabilities that can treat the solvers as black-box. This means that we can solve both the governing equations and sensitivity response using the same code. Second, a robust analysis technique for the coupled FSI system based on IB method is formulated. The current approach of IB is not suited for the sensitivity analysis due to the discontinuities in its formulation. This will be explained in more details in the following Chapters.

% ======================================================================================
\section{Sensitivity Analysis Literature}
