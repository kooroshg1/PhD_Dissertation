\chapter{Introduction}
\section{Motivation}
Fluid-structure interaction (FSI) problems play important role in many scientific and engineering fields, such as automotive, aerospace, and biomedical industry. In all these cases, the engineers try to design the systems based on the coupling between the fluid and structural disciplines. Despite the application of FSI problems in different industries, a comprehensive study of such problems remains a challenge due to their strong nonlinearity and multidisciplinary nature. For most FSI problems, analytical solutions to are not available, and laboratory experiments are limited in scope. Therefore, to get more insight in the physics involved in the complex interaction between fluids and solids, numerical simulations are used. Nevertheless, the prohibitive amount of computations has been one of the major issues in the design of coupled FSI systems. 

On of the root causes of this exesive computational cost is the computational mesh used to representing the complex and moving shape of the structure. The conventional approach discretize the domain with mesh that conforms to the boundary of the solid region. Hence, when the solid boundary moves, the mesh needs to be updated/deformed. The methods that are available add computational time to an already expensive computation. This will become a challenge specially When intended to design optimization for the large scale problems, because it is impractical to employ convectional finite difference sensitivity to perform shape design optimization of the coupled FSI problems.