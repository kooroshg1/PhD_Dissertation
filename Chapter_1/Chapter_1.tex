\chapter{Introduction}
\section{Motivation}
Fluid-structure interaction (FSI) problems play important role in many scientific and engineering fields, such as automotive, aerospace, and biomedical industry. Despite the wide application, a comprehensive study of FSI systems still remains a challenge due to their strong nonlinearity and multidisciplinary nature. For most FSI problems, analytical solutions to are not available, and physical experiments are limited in scope. Therefore, to get more insight in the physics involved in the complex interaction between fluids and solids, numerical simulations are used. The numerical solutions are conducted based on Computational Fluid Dynamics (CFD) models for the flow and Finite Element Analysis (FEA) for the structural response. Nevertheless, the prohibitive amount of computations has been one of the major issues in the design optimization of such coupled multidisciplinary systems. The other major bottle neck is generating an appropriate computational domain that represents the fluid and solid regions. For complex and moving geometries in fluid, this is impossible to automate. Moreover, to design and optimization of these systems using gradient based optimization method, sensitivity analysis is required. Although there are analytical techniques for efficient and accurate calculation of sensitivities, they are not yet implemented in commercial CFD packages. Therefore, most gradient optimization techniques relay on finite difference method for sensitivity calculation.

The motivation for the research proposed in this document is in two areas. First, a robust analysis technique for the coupled FSI system requires computational fluid meshes that do not depend on the shape of the solid boundaries. This allows continuous change in the shape of the immersed solid structure without modifying the fluid domain .

%Due to the expensive simulations of the FSI problems, the optimization methods based on large number of simulations such as Genetic Algorithms are impractical. The gradient based optimization is a good choice for design optimization of such systems. However, these techniques require subroutines for calculating the derivative of the objective function or as known in the optimization community, the sensitivities. The finite difference method, is the easiest technique for calculating the sensitivities. However, it has high cost. Moreover, convergence studies need to be done to select the appropriate step size. Therefore, analytical techniques

%On of the root causes of this exesive computational cost is the computational mesh used to representing the complex and moving shape of the structure. The conventional approach discretize the domain with mesh that conforms to the boundary of the solid region. Hence, when the solid boundary moves, the mesh needs to be updated/deformed. The methods that are available add computational time to an already expensive computation. This will become a challenge specially When intended to design optimization for the large scale problems, because it is impractical to employ convectional finite difference sensitivity to perform shape design optimization of the coupled FSI problems.